% !TEX root = ./manual.tex
%%
% configurations
% @author Thomas Lehmann
%
\section{Konfiguration des Templates}\label{sec:configuraitons}
Das Template muss für die eigene Arbeit angepasst werden. Alle Einstellungen sind in der Datei \texttt{configuration/configuration.tex} vorzunehmen. Die Einstellungsmöglichkeiten und ihre Auswirkungen sind im folgenden aufgelistet.

\paragraph{Betreuer}\index{Einstellungen!firstSupervisor}\index{Einstellungen!secondSupervisor} Der Betreuer (Erstgutachter) und der Zweitgutachter werden über den Parameter der Kommandos \texttt{\textbackslash firstSupervisor\{\}} und entsprechend mit dem Kommando \texttt{\textbackslash secondSupervisor\{\}} eingestellt. Hier sind die beiden Default-Namen gegen die Namen der Gutachter auszutauschen.

\paragraph{Titel der Arbeit}\index{Einstellungen!thesisTitle} Der eigentliche Titel der Arbeit wird im Parameter des Kommandos \texttt{\textbackslash thesisTitle\{\}} angegeben. Ist der Titel länger und bildet drei Zeilen, so muss die Formatierung der Titelseite angepasst werden. Der Titel ist für die Abstract-Page auch in Englisch über das Kommando \texttt{\textbackslash thesisTitleEN\{\}} anzugeben.

\paragraph{Keywords}\index{Einstellungen!Keywords} Für die Arbeit sind Schlüsselwörter in Englisch über Parameter des Kommandos \texttt{\textbackslash keywordsEN\{\}} und in Deutsch über das Kommando \texttt{\textbackslash keywordsDE\{\}} anzugeben. Die Begriffe werden auf der Abstact-Page eingefügt.

\paragraph{Abstract}\index{Einstellungen!Abstract} Für die Arbeit ist eine Zusammenfassung in Deutsch und in Englisch zu erstellen. Mit dem Kommando \texttt{\textbackslash abstractEN\{\}} ist die englische Zusammenfassung und mit dem Kommando \texttt{\textbackslash abstractDE\{\}} ist die
deutsche Zusammenfassung anzugeben.

\paragraph{Autor}\index{Einstellungen!thesisAuthor} Der Autor wird im Parameter des Kommandos \texttt{\textbackslash thesisAutor\{\}} angegeben. Hier ist entsprechend der eigene Name einzusetzen.

\paragraph{Abgabedatum}\index{Einstellungen!submissiondate} Hier ist das Datum der Abgabe, so wie es im Dokument stehen soll, als Text im Parameter des Kommandos \texttt{\textbackslash submissionDate\{\}} anzugeben.

\paragraph{Non-Disclosure Agreement/Geheimhaltung}\index{Einstellungen!NDA} Ist die Arbeit Geheim, d.~h. es besteht ein Geheimhaltungsvertrag (NDA), so ist das Kommando \texttt{\textbackslash NDA\{yes\}} auszukommentieren. Ein entsprechender Vermerk erscheint dann auf dem Titelblatt. Ist die Arbeit nicht geheim, so ist \texttt{\textbackslash NDA\{no\}} auszukommentieren.

\paragraph{Abbildungsverzeichnis} Wird kein Abbildungsverzeichnis benötigt, so ist der Parameter für das Kommando \texttt{\textbackslash ListOfFigures} auf \texttt{no} zu setzen.

\paragraph{Tabellenverzeichnis} Wird kein Tabellenverzeichnis benötigt, so ist der Parameter für das Kommando \texttt{\textbackslash ListOfTables} auf \texttt{no} zu setzen.

\paragraph{Akronymverzeichnis} Wird kein Akronymverzeichnis benötigt, so ist der Parameter für das Kommando \texttt{\textbackslash ListOfAccronym} auf \texttt{no} zu setzen. Beim Erstellen des PDFs muss als zusätzlicher Kompileschritt \texttt{makeglossaries} ausgeführt werden.

\paragraph{Symbolverzeichnis} Wird kein Symbolverzeichnis benötigt, so ist der Parameter für das Kommando \texttt{\textbackslash ListOfSymbols} auf \texttt{no} zu setzen. Beim Erstellen des PDFs muss als zusätzlicher Kompileschritt \texttt{makeglossaries} ausgeführt werden.

\paragraph{Glossary} Wird kein Glossary an Ende benötigt, so ist der Parameter für das Kommando \texttt{\textbackslash Glossary} auf \texttt{no} zu setzen. Beim Erstellen des PDFs muss als zusätzlicher Kompileschritt \texttt{makeglossaries} ausgeführt werden.

\paragraph{Studiengang} Der Studiengang ist über das Kommando
\texttt{\textbackslash studycourse\{\}} mit entsprechendem Parameter
auszuwählen. Als Parameter sind zulässig:

\begin{center}
  \begin {tabular}{ll}
    Parameter & Studiengang \\
    \hline
    \texttt{TI} & Bachelor Technische Informatik \\
    \texttt{AI} & Bachelor Angewandte Informatik \\
    \texttt{WI} & Bachelor Wirtschaftsinformatik \\
    \texttt{EI} & Bachelor Elektro- und Informationstechnik \\
    \texttt{BMT} & Bachelor Mechatronik \\
    \texttt{MAI} & Master Informatik \\
    \texttt{MIK} & Master Informations- und Kommunikationstechnik \\
    \texttt{MA} & Master Automatisierung \\
  \end{tabular}
\end{center}

\section{Konfiguration im \LaTeX -Code}\label{sec:codechanges}
Für einige Anpassungen müssen direkt eine Änderungen im \LaTeX -Code vorgenommen werden. Soll eine Liste von Listings eingefügt werden, so ist die Zeile in der Datei \texttt{thesis.tex} auszukommentieren und das Paket \texttt{listings} in der Paketlist hinzuzufügen.
